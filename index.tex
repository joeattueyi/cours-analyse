\documentclass[10pt,letterpaper]{book}
\usepackage[utf8]{inputenc}
\usepackage[english]{babel}
\usepackage{amsmath}
\usepackage{amsfonts}
\usepackage{enumerate}
\usepackage{amssymb}
\usepackage{amsthm}
\usepackage{graphicx}
\usepackage{lmodern}
\usepackage{graphicx}
\usepackage{color}
\usepackage{hyperref}
\usepackage{chngcntr}
\usepackage{titlesec}

\title{A Course in Analysis}
\author{Camille Jordan}

\hypersetup{
    colorlinks=true,
    linktoc=all,     
    linkcolor=blue,
}

% \setlength{\parindent}{0pt}
% \setlength{\parskip}{10pt plus5pt minus5pt}

\graphicspath{ {img/} }


\newcommand{\C}{\mathbb C}
\newcommand{\R}{\mathbb R}
\newcommand{\N}{\mathbb N}
\newcommand{\pd}[2]{\partial_{#2}{#1}}
\newcommand{\fpd}[2]{\frac{\partial #1}{\partial #2}}
\newcommand{\Aut}{\mathrm{Aut\kern 2pt}}
\newcommand{\End}{\mathrm{End\kern 2pt}}
\newcommand{\inv}[1]{{#1}^{-1}}
\newcommand{\centerimage}[2]{\begin{center}\includegraphics[#1]{#2}\end{center}}
\newcommand{\norm}[1]{\left\|{#1}\right\|}
\newcommand{\supnorm}[1]{\left\|{#1}\right\|_\infty}
\newcommand{\seq}[2]{\left({#1}\right)_{#2}}
\newcommand{\overbar}[1]{\mkern 1.5mu\overline{\mkern-1.5mu#1\mkern-1.5mu}\mkern 1.5mu}
\newcommand{\Hol}[1]{\mathcal{H}(#1)}

\renewcommand\qedsymbol{$\blacksquare$}
\renewcommand\epsilon{\varepsilon}

\theoremstyle{definition}
\newtheorem{theorem}{Theorem}
\newtheorem{lemma}{Lemma}
\newtheorem{definition}{Definition}

\counterwithout{paragraph}{subsubsection}
\renewcommand{\theparagraph}{\arabic{paragraph}}
\setcounter{secnumdepth}{4}
\titleformat{\paragraph}[runin]{\normalfont\bfseries}{\theparagraph.}{\wordsep}{}
\titlespacing{\paragraph}{0pt}{3.25ex plus 1ex minus .2ex}{\wordsep}

\begin{document}

\maketitle
\tableofcontents

% [Original: Page 1]

\part{Differential Calculus}

\chapter{Real Variables}

\section{Limits}

\paragraph{} 
The main objective of Arithmetic is the study of whole numbers.

To solve, in every case, first order equations, we have to extend this primitive notion by introducing negative and rational numbers.

The solution of equations of degree $> 1$ requires new generalizations. The principles on which these are based are closely related to those of infinitesimal calculus, so we briefly touch on them.

% [Original: Page 2]

\paragraph{Irrational numbers} Let $A$ and $B$ be two sets of rational numbers having the following properties
\begin{enumerate}
\item All numbers in $B$ are greater then all numbers in $A$
\item For any given positive number $\epsilon$ there are $a$ in $A$ and $b$ in $B$ such that
\[
  b - a < \epsilon
\]
\end{enumerate}
Given these hypostheses, the rational numbers can be partitioned into three classes: the first, $\mathcal{A}$, contains all numbers smaller then any number in $A$; the second, $\mathcal B$, all numbers greater then any number in $B$; the third, $\mathcal C$, those that are neither smaller then a number in $A$, nor greater then a number in $B$.

It is impossible for this class $\mathcal C$ to contain two different numbers $c$ and $c' < c$. In fact for any choice of $a$ and $b$,
\[
  a\leq c', b\geq c\quad\mbox{we get}\quad b-a\geq c-c'
\]
contrary to out second hypotesis.

It might contain a unique number $c$ (this is the case, for example, if we take for $A$ the set of numbers $< c$, and for $B$ the set of numbers $> c$).

But it might also happen that it doesn't contain any (we know this to be the case if we take for $A$ all the even convergents, and for $B$ all the odd convergents of an infinite continued fraction).

We will make this distinction disappear, and say that, in the second case, there still exists a number $c$ greater then all numbers in $\mathcal A$ and smaller then those in $\mathcal B$, but that such a number is \textit{irrational}.

The set of numbers so obtained, both rational and irrational, constitutes the set of \textit{real} numbers.

% [Original: Page 3]

Each of these numbers is identified, as seen, by all rational numbers that are smaller and greater then it.

\paragraph{}


\begin{thebibliography}{9}
\addcontentsline{toc}{chapter}{Bibliography}

\end{thebibliography}

\end{document}