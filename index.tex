\documentclass[10pt,letterpaper]{book}
\usepackage[utf8]{inputenc}
\usepackage[english]{babel}
\usepackage{amsmath}
\usepackage{amsfonts}
\usepackage{enumerate}
\usepackage{amssymb}
\usepackage{amsthm}
\usepackage{graphicx}
\usepackage{lmodern}
\usepackage{graphicx}
\usepackage{color}
\usepackage{hyperref}
\usepackage{chngcntr}
\usepackage{titlesec}

\title{A Course in Analysis}
\author{Camille Jordan}

\hypersetup{
    colorlinks=true,
    linktoc=all,     
    linkcolor=blue,
}

% \setlength{\parindent}{0pt}
% \setlength{\parskip}{10pt plus5pt minus5pt}

\graphicspath{ {img/} }


\newcommand{\C}{\mathbb C}
\newcommand{\R}{\mathbb R}
\newcommand{\N}{\mathbb N}
\newcommand{\pd}[2]{\partial_{#2}{#1}}
\newcommand{\fpd}[2]{\frac{\partial #1}{\partial #2}}
\newcommand{\Aut}{\mathrm{Aut\kern 2pt}}
\newcommand{\End}{\mathrm{End\kern 2pt}}
\newcommand{\inv}[1]{{#1}^{-1}}
\newcommand{\centerimage}[2]{\begin{center}\includegraphics[#1]{#2}\end{center}}
\newcommand{\norm}[1]{\left\|{#1}\right\|}
\newcommand{\supnorm}[1]{\left\|{#1}\right\|_\infty}
\newcommand{\seq}[2]{\left({#1}\right)_{#2}}
\newcommand{\overbar}[1]{\mkern 1.5mu\overline{\mkern-1.5mu#1\mkern-1.5mu}\mkern 1.5mu}
\newcommand{\Hol}[1]{\mathcal{H}(#1)}

\renewcommand\qedsymbol{$\blacksquare$}
\renewcommand\epsilon{\varepsilon}

\theoremstyle{definition}
\newtheorem{theorem}{Theorem}
\newtheorem{lemma}{Lemma}
\newtheorem{definition}{Definition}

\counterwithout{paragraph}{subsubsection}
\renewcommand{\theparagraph}{\arabic{paragraph}}
\setcounter{secnumdepth}{4}
\titleformat{\paragraph}[runin]{\normalfont\bfseries}{\theparagraph.}{\wordsep}{}
\titlespacing{\paragraph}{0pt}{3.25ex plus 1ex minus .2ex}{\wordsep}

\begin{document}

\maketitle
\tableofcontents

% [Original: Page 1]

\part{Differential Calculus}

\chapter{Real Variables}

\section{Limits}

\paragraph{} 
The main objective of Arithmetic is the study of whole numbers.

To solve, in every case, first order equations, we have to extend this primitive notion by introducing negative and rational numbers.

The solution of equations of degree $> 1$ requires new generalizations. The principles on which these are based are closely related to those of infinitesimal calculus, so we briefly touch on them.

% [Original: Page 2]

\paragraph{Irrational numbers}\label{par:irrational-numbers} Let $A$ and $B$ be two sets of rational numbers having the following properties
\begin{enumerate}
\item All numbers in $B$ are greater than all numbers in $A$
\item For any given positive number $\epsilon$ there are $a$ in $A$ and $b$ in $B$ such that
\[
  b - a < \epsilon
\]
\end{enumerate}
Given these hypostheses, the rational numbers can be partitioned into three classes: the first, $\mathcal{A}$, contains all numbers less than any number in $A$; the second, $\mathcal B$, all numbers greater than any number in $B$; the third, $\mathcal C$, those that are neither less than a number in $A$, nor greater than a number in $B$.

It is impossible for this class $\mathcal C$ to contain two different numbers $c$ and $c' < c$. In fact for any choice of $a$ and $b$,
\[
  a\leq c', b\geq c\quad\mbox{we get}\quad b-a\geq c-c'
\]
contrary to out second hypotesis.

It might contain a unique number $c$ (this is the case, for example, if we take for $A$ the set of numbers $< c$, and for $B$ the set of numbers $> c$).

But it might also happen that it doesn't contain any (we know this to be the case if we take for $A$ all the even convergents, and for $B$ all the odd convergents of an infinite continued fraction).

We will make this distinction disappear, and say that, in the second case, there still exists a number $c$ greater than all numbers in $\mathcal A$ and less than those in $\mathcal B$, but that such a number is \textit{irrational}.

The set of numbers so obtained, both rational and irrational, constitutes the set of \textit{real} numbers.

% [Original: Page 3]

Each of these numbers is identified, as seen, by all rational numbers that are less and greater than it.

\paragraph{} If a rational number $r$ is less (resp. greater) than a real number $c$, then it is clear, after our definitions, that all rational numbers $r'$ less (resp. greater) than $r$ will also be less (resp. greater) than $c$.

Moreover we can show that \textit{there is an infinity of rational numbers between $r$ and $c$}.

In fact, if $c$ is rational, all numbers of the form $r+m(c-r)$ with $m$ any rational number between $0$ and $1$, will satify this condition.

If $c$ is irrational, lets suppose, for example, $c > r$. Let $A, B$ be two sets of rational numbers that determine $c$. All rational numbers belong, by hypothesis, to one of the classes $\mathcal A, \mathcal B$ defined above.

The number $r < c$ belongs to $\mathcal A$. It's therefore less than a number $a$ in $A$. This number $a$ being less than all those in $B$ cannot be in $\mathcal B$ and so it belongs to $\mathcal A$. It will therefore be less than another number $a_1$ in $A$. continuing like this, we obtain an infinite increasing sequence of rational numbers $r, a_1, \dots$, all less than $c$.

We will say that a real number $c$ is greater than another real number $c'$ if there exists a rational number $r$ which is $<c$ and $> c'$. In this case, all rational numbers between $c$ and $r$ or between $r$ and $c'$, of which there is an infinity, enjoy the same property.

It's clear that if $c> c'$ and $c' > c''$ then $c > c''$.

To conclude, \textit{between any two rational number $r$ and $r+\epsilon$, there exists an irrational number} (and, therefore, an infinity of them).

% [Original: Page 4]

Let, in fact, $c$ be an irrational number defined by two sets of rational numbers $A$ and $B$. We can find in them two numbers $a$ and $b$ with a difference $<\epsilon$, and such that $c$ lays between $a$ and $b$.

Let $A_1$,$B_1$ be the two sets obtained by adding $r-a$ to all the numbers in $A$ and in $B$. It's evident that these two new sets have the properties required to defined a new irrational number laying between $r$ and $r+\epsilon$.

A real number $c$ will be called \textit{positive} if it's $> 0$ (in which case it will still be greater than an infinity of positive rational numbers), \textit{negative} it it's $< 0$.

The inequalities
\[
  b-a > 0,\quad b-a < \epsilon
\]
can be written
\[
  (-a) - (-b) > 0,\quad (-a)-(-b) < \epsilon,
\]
and we see that to any real number $c$, defined by a set of numbers $a$ and one of numbers $b$, corresponds another number of opposite sign, defined by the set of numbers $-b$ and that of numbers $-a$. We will indicate such a number by $-c$. We have, after this definition,
\[
  -(-c) = c.
\]

\paragraph{} It remains to generalize the definition of arithmetic operations, to make them applicable to all real numbers.

Let $c, c'$ be two such numbers, identified respectively by the sets $A, B; A', B'$. Let $a,b;a', b'$ be numbers taken, respectively, from them; we have
\[
  b+b' > a+a'.
\]
On the other hand, we can choose $a,b,a',b'$ so that
\[
  b-a < \frac \epsilon 2,\quad b'-a' < \frac \epsilon 2
\]
which implies that
\[
  b + b' - (a + a')< \epsilon
\]

% [Original: Page 5]

The set of numbers $a+a'$ and that of numbers $b+b'$ therefore define a real number, which we will denote $c+c'$.

Likewise, the set of numbers $b-a'$ and that of numbers $a-b'$ will define a number, which we will denote $c-c'$.

Clearly, after this definition,
\[
  c + c' = c' + c,\quad c-c'=c+(-c')
\]

Subtraction is the inverse operation of addition, so that the number $(c-c')+c'=c_1$ is none other then $c$. To show this, we will prove that all rational numbers $>c$ are $>c_1$, and reciprocally.

The numbers $> c$ are those greater than one of the numbers $b$, the numbers $> c_1$ are those greater than one of the numbers $b-a'+b'$.

Now, all numbers $x>b-a'+b'$ are also $>b$, because $b'> a'$.

On the other hand, if $x > c$, we can find between $x$ and $c$  an infinity of rational numbers $> c$, Let $x-\epsilon = b$ be one of them, then
\[
  x > b-a'+b'
\]
if we choose $a'$ and $b'$ so that $b'-a'< \epsilon$.

\paragraph{} To define multiplication and division, we first assume $c$ and $c'$ to be positive.

Let $\alpha, \alpha'$ be two abitrary fixed positive numbers contained in the sets $A, A'$; $\beta, \beta'$ two fixed numbers (necessarily positive) chosen from the sets $B, B'$; $a, b, a', b'$ positive numbers from these same sets; we will always have
\[
  bb' > aa',\quad \frac b {a'} > \frac a {b'}
\]

% [Original: Page 6]

We can choose these numbers so that
\[
  b-a < \delta,\quad b'-a' < \delta
\]
where $\delta$ is a quantity to be determined later on.
We can also suppose that $a$ and $b$ are contained in the interval from $\alpha$ to $\beta$, $a'$ and $b'$ in the interval from $\alpha'$ to $\beta'$. In fact, $a$, for example, is certainly $<\beta$. If it's $<\alpha$, then $b-\alpha < b-a < \delta$. We can then substitute $\alpha$ for $a$ and maintain all the required inequalities.

Given this, we will have
\begin{align*}
  bb'-aa' &< (a+\delta)(a'+\delta) - aa'\\
    &<(a+a')\delta + \delta^2 < (\beta+\beta')\delta + \delta^2;\\
 \frac{b}{a'} - \frac{a}{b'} &= \frac{bb'-aa'}{a'b'} < 
   \frac{(\beta + \beta')\delta + \delta^2}{\alpha'^2}    
\end{align*}
If we choose $\delta$ so that it satisfies the inequalities
\[
  \delta < 1,\quad 
  \delta < \frac{\epsilon}{\beta+\beta'+1},\quad
  \delta < \frac{\alpha'^2\epsilon}{\beta+\beta+1}
\]
we will have
\begin{align*}
  bb'-aa'&<(\beta + \beta' + 1)\delta < \epsilon\\
  \frac{b}{a'}-\frac{a}{b'}&<
  \frac{(\beta+\beta+1)\delta}{\alpha'^2} < \epsilon
\end{align*}

The set of number $aa'$, together with that of numbers $bb'$, and the set of numbers $\frac{a}{b'}$, together with that of numbers $\frac{b}{a'}$, define therefore two real numbers, which we will refer to as $cc'$ and $\frac{c}{c'}$.

The division so defined is the inverse of the multiplication. To establish this, we must establish the equality of the numbers $\frac{c}{c'}c=c_1$ and $c$, by showing that the rational numbers greater than one of them are greater than the other, and vice-versa.

Let $x$ be a number $>c_1$. It will be, by definition, greater than
% [Original: Page 7]
one of the numbers $\frac{b}{a'}b'$, and, a fortiori, greater than the number $b$; it will therefore be $> c$.

Reciprocally, if $x > c$, there will exist another rational number $x-\epsilon$ still greater then $c$, which means greater then one of the numbers $b$. Now we can determine, for any $\delta, a'$ and $b'$ such that $b'<a'+\delta$, that
\[
  \frac{b}{a'}b'<b+\frac{b\delta}{a'}<b+\frac{\beta\delta}{\alpha'}.
\]
If we choose $\delta$ less than $\frac{\alpha'\epsilon}{\beta'}$, we will have $x>\frac{b}{a'}b'$. Therefore $x > c_1$.

To extend the definition of the multiplication and the division to the case where one the factors, or both, are negative, we apply the rule of signs. To conclude we state without proof that a product is zero if one of the factors is.

We see, without trouble, that the operations so generalized, applied to rational numbers, will lead to the same results of the normal operations, and that the algebraic rules of computation remain valid without any change.

\paragraph{} We call \textit{absolute value} or \textit{module} of a real number $c$ the number itself, if it's positive, the number $-c$ if $c$ is negative. The module is often denoted by $|c|$. 

We clearly have
\begin{align*}
  |a||b| &= |ab| \\
  |a|-|b|-|c| &\leq a \pm b \pm c \leq |a|+|b|+|c|
\end{align*}

\paragraph{} Let $A,B$ be two sets of real numbers such that: first, all numbers in $B$ are greater then all numbers in $A$; second, we can find in $A$ and $B$ two numbers $a$ and $b$ such that $b-a<\epsilon$. Then there exists a unique real number $c$ such that we have
\[
  b\geq c \geq a
\]
for all $a$ in $A$ and $b$ in $B$.

In fact, let $A'$ be the set of rational numbers
% [Original: page 8]
less than at least one of the numbers in $A$; $B'$ the set of rational numbers greater than at least one of the numbers in $B$. The numbers in $B'$ will be greater then the ones in $A'$.

Moreover, take from $A$ and $B$ two numbers $a$ and $b$ such that $b-a<\frac \epsilon 3$. We can find two rational numbers around $a$ with a difference $<\frac \epsilon 3$ of which the smaller $a'$ belongs to $A'$, so we have
\[
  a - a' < \frac \epsilon 3.
\]
Likewise we can find in $B'$ a number $b'$ such that $b'-b$ is $<\frac\epsilon 3$. We have, therefore,
\[
  b' - a' < \epsilon.
\]

The sets $A',B'$, made of rational numbers, determine a real number $c$ greater than all the $a'$ and less than all the $b'$. This number satisfies the conditions required. In fact, if it's less than a number $a$ in $A$, then there would exist between $a$ and $c$ some rational number $a'$ greater than $c$, contrary to the definition of $c$. If it's greater than a number $b$ in $B$, we would arrive to an analogous contraddiction.

We can see, like in paragraph \ref{par:irrational-numbers}, that such a number $c$ is unique.

\paragraph{Limits} Let $x$ be a variable quantity, to which we give in succession an infinite sequence of values $x_1,\dots,x_n,\dots$. We say that the sequence $x_1,\dots,x_n,\dots$, or, in a more algebraic way, the variable $x$ \textit{tends} or \textit{converges} toward a \textit{limit} $c$ if, for all values of a positive quantity $\epsilon$, we can find an integer $\nu$ such that we have
\[
  |x_n-c| < \epsilon
\]
for all values of $n$ greater than $\nu$.

% [Original: page 9]

The variable $x$ cannot tend at the same time to two different limits $c$ and $c'$; because we would have
\[
  c'-c = (x_n-c)-(x_n-c'),
\]
from which
\[
  |c' - c| \leq |x_n-c| + |x_n-c'|.
\]
Therefore, regardless of $n$, at least one of the two modules
\[
  |x_n-c|,\quad|x_n-c'|
\]
would be at least equal to $\frac 1 2 |c'-c|$.

It is important to transform the preceeding definition, of a way to prove the existence of a limit, even when we are unable to determine it.

\begin{thebibliography}{9}
\addcontentsline{toc}{chapter}{Bibliography}

\end{thebibliography}

\end{document}