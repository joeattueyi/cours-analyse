\documentclass[10pt,letterpaper]{book}
\usepackage[utf8]{inputenc}
\usepackage[english]{babel}
\usepackage{amsmath}
\usepackage{amsfonts}
\usepackage{enumerate}
\usepackage{amssymb}
\usepackage{amsthm}
\usepackage{graphicx}
\usepackage{lmodern}
\usepackage{graphicx}
\usepackage{color}
\usepackage{hyperref}
\usepackage{chngcntr}
\usepackage{titlesec}

\title{A Course in Analysis}
\author{Camille Jordan}

\hypersetup{
    colorlinks=true,
    linktoc=all,     
    linkcolor=blue,
}

% \setlength{\parindent}{0pt}
% \setlength{\parskip}{10pt plus5pt minus5pt}

\graphicspath{ {img/} }


\newcommand{\C}{\mathbb C}
\newcommand{\R}{\mathbb R}
\newcommand{\N}{\mathbb N}
\newcommand{\pd}[2]{\partial_{#2}{#1}}
\newcommand{\fpd}[2]{\frac{\partial #1}{\partial #2}}
\newcommand{\Aut}{\mathrm{Aut\kern 2pt}}
\newcommand{\End}{\mathrm{End\kern 2pt}}
\newcommand{\inv}[1]{{#1}^{-1}}
\newcommand{\centerimage}[2]{\begin{center}\includegraphics[#1]{#2}\end{center}}
\newcommand{\norm}[1]{\left\|{#1}\right\|}
\newcommand{\supnorm}[1]{\left\|{#1}\right\|_\infty}
\newcommand{\seq}[2]{\left({#1}\right)_{#2}}
\newcommand{\overbar}[1]{\mkern 1.5mu\overline{\mkern-1.5mu#1\mkern-1.5mu}\mkern 1.5mu}
\newcommand{\Hol}[1]{\mathcal{H}(#1)}

\renewcommand\qedsymbol{$\blacksquare$}
\renewcommand\epsilon{\varepsilon}

\theoremstyle{definition}
\newtheorem{theorem}{Theorem}
\newtheorem{lemma}{Lemma}
\newtheorem{definition}{Definition}

\counterwithout{paragraph}{subsubsection}
\renewcommand{\theparagraph}{\arabic{paragraph}}
\setcounter{secnumdepth}{4}
\titleformat{\paragraph}[runin]{\normalfont\bfseries}{\theparagraph.}{\wordsep}{}
\titlespacing{\paragraph}{0pt}{3.25ex plus 1ex minus .2ex}{\wordsep}

\begin{document}

\maketitle
\tableofcontents

% [Original: Page 1]

\part{Differential Calculus}

\chapter{Real Variables}

\section{Limits}

\paragraph{} 
The main objective of Arithmetic is the study of whole numbers.

To solve all possible first order equations we have to extend this primitive notion by introducing negative and rational numbers.

The solution of equations of degree $> 1$ requires new generalizations. The principles on which these are based are closely related to those of infinitesimal calculus, so we briefly touch on them.

% [Original: Page 2]

\paragraph{Irrational numbers}\label{par:irrational-numbers} Let $A$ and $B$ be two sets of rational numbers having the following properties
\begin{enumerate}
\item Any number in $B$ is greater than any number in $A$
\item For any given positive number $\epsilon$ there are $a$ in $A$ and $b$ in $B$ such that
\[
  b - a < \epsilon
\]
\end{enumerate}
Given these hypostheses, the rational numbers can be partitioned into three classes: the first, $\mathcal{A}$, contains all numbers less than any number in $A$; the second, $\mathcal B$, all numbers greater than any number in $B$; the third, $\mathcal C$, those that are neither less than a number in $A$, nor greater than a number in $B$.

It is impossible for this class $\mathcal C$ to contain two different numbers $c$ and $c' < c$. In fact for any choice of $a$ and $b$,
\[
  a\leq c', b\geq c\quad\mbox{we get}\quad b-a\geq c-c'
\]
contrary to our second hypotesis.

It might contain a unique number $c$ (this is the case, for example, if we take for $A$ the set of numbers $< c$, and for $B$ the set of numbers $> c$).

But it might also happen that it doesn't contain any (we know this to be the case if we take for $A$ all the even convergents, and for $B$ all the odd convergents of an infinite continued fraction).

We will make this distinction disappear, and say that, in the second case, there still exists a number $c$ greater than all numbers in $\mathcal A$ and less than those in $\mathcal B$, but that such a number is \textit{irrational}.

The set of numbers so obtained, both rational and irrational, constitutes the set of \textit{real} numbers.

% [Original: Page 3]

Each of these numbers is identified, as seen, by all rational numbers that are less and greater than it.

\paragraph{} If a rational number $r$ is less (resp. greater) than a real number $c$, then it is clear, after our definitions, that all rational numbers $r'$ less (resp. greater) than $r$ will also be less (resp. greater) than $c$.

Moreover we can show that \textit{there is an infinity of rational numbers between $r$ and $c$}.

In fact, if $c$ is rational, all numbers of the form $r+m(c-r)$ with $m$ any rational number between $0$ and $1$, will satify this condition.

If $c$ is irrational, lets suppose, for example, that $c > r$. Let $A, B$ be two sets of rational numbers that determine $c$. All rational numbers belong, by hypothesis, to one of the classes $\mathcal A, \mathcal B$ defined above.

The number $r < c$ belongs to $\mathcal A$. It's therefore less than a number $a$ in $A$. This number $a$ being less than all those in $B$ cannot be in $\mathcal B$ and so it belongs to $\mathcal A$. It will therefore be less than another number $a_1$ in $A$. continuing like this, we obtain an infinite increasing sequence of rational numbers $r, a_1, \dots$, all less than $c$.

We will say that a real number $c$ is greater than another real number $c'$ if there exists a rational number $r$ which is $<c$ and $> c'$. In this case, all rational numbers between $c$ and $r$ or between $r$ and $c'$, of which there is an infinity, enjoy the same property.

It's clear that if $c> c'$ and $c' > c''$ then $c > c''$.

Finally, \textit{between any two rational number $r$ and $r+\epsilon$, there exists an irrational number} (and, therefore, an infinity of them).

% [Original: Page 4]

In fact, let $c$ be an irrational number defined by two sets of rational numbers $A$ and $B$. We can find in them two numbers $a$ and $b$ with a difference $<\epsilon$, and such that $c$ lays between $a$ and $b$.

Let $A_1$,$B_1$ be the two sets obtained by adding $r-a$ to all the numbers in $A$ and in $B$. It's evident that these two new sets have the properties required to defined a new irrational number laying between $r$ and $r+\epsilon$.

A real number $c$ will be called \textit{positive} if it's $> 0$ (in which case it will still be greater than an infinity of positive rational numbers), \textit{negative} it it's $< 0$.

The inequalities
\[
  b-a > 0,\quad b-a < \epsilon
\]
can be written
\[
  (-a) - (-b) > 0,\quad (-a)-(-b) < \epsilon,
\]
and we see that to any real number $c$, defined by a set of numbers $a$ and one of numbers $b$, corresponds another number of opposite sign, defined by the set of numbers $-b$ and that of numbers $-a$. We will indicate such a number by $-c$. We have, after this definition,
\[
  -(-c) = c.
\]

\paragraph{} It remains to generalize the definition of arithmetic operations, to make them applicable to all real numbers.

Let $c, c'$ be two such numbers, identified respectively by the sets $A, B; A', B'$. Let $a,b;a', b'$ be numbers taken, respectively, from them; we have
\[
  b+b' > a+a'.
\]
On the other hand, we can choose $a,b,a',b'$ so that
\[
  b-a < \frac \epsilon 2,\quad b'-a' < \frac \epsilon 2
\]
which implies that
\[
  b + b' - (a + a')< \epsilon
\]

% [Original: Page 5]

The set of numbers $a+a'$ and that of numbers $b+b'$ therefore define a real number, which we will denote $c+c'$.

Likewise, the set of numbers $b-a'$ and that of numbers $a-b'$ will define a number, which we will denote $c-c'$.

Clearly, after this definition,
\[
  c + c' = c' + c,\quad c-c'=c+(-c')
\]

Subtraction is the inverse operation of addition, so that the number $(c-c')+c'=c_1$ is none other then $c$. To show this, we will prove that all rational numbers $>c$ are $>c_1$, and reciprocally.

The numbers $> c$ are those greater than one of the numbers $b$, the numbers $> c_1$ are those greater than one of the numbers $b-a'+b'$.

Now, all numbers $x>b-a'+b'$ are also $>b$, because $b'> a'$.

On the other hand, if $x > c$, we can find between $x$ and $c$  an infinity of rational numbers $> c$, Let $x-\epsilon = b$ be one of them, then
\[
  x > b-a'+b'
\]
if we choose $a'$ and $b'$ so that $b'-a'< \epsilon$.

\paragraph{} To define multiplication and division, we first assume $c$ and $c'$ to be positive.

Let $\alpha, \alpha'$ be two abitrary fixed positive numbers contained in the sets $A, A'$; $\beta, \beta'$ two fixed numbers (necessarily positive) chosen from the sets $B, B'$; $a, b, a', b'$ positive numbers from these same sets; we will always have
\[
  bb' > aa',\quad \frac b {a'} > \frac a {b'}
\]

% [Original: Page 6]

We can choose these numbers so that
\[
  b-a < \delta,\quad b'-a' < \delta
\]
where $\delta$ is a quantity to be determined later on.
We can also suppose that $a$ and $b$ are contained in the interval from $\alpha$ to $\beta$, $a'$ and $b'$ in the interval from $\alpha'$ to $\beta'$. In fact, $a$, for example, is certainly $<\beta$. If it's $<\alpha$, then $b-\alpha < b-a < \delta$. We can then substitute $\alpha$ for $a$ and maintain all the required inequalities.

Given this, we will have
\begin{align*}
  bb'-aa' &< (a+\delta)(a'+\delta) - aa'\\
    &<(a+a')\delta + \delta^2 < (\beta+\beta')\delta + \delta^2;\\
 \frac{b}{a'} - \frac{a}{b'} &= \frac{bb'-aa'}{a'b'} < 
   \frac{(\beta + \beta')\delta + \delta^2}{\alpha'^2}    
\end{align*}
If we choose $\delta$ so that it satisfies the inequalities
\[
  \delta < 1,\quad 
  \delta < \frac{\epsilon}{\beta+\beta'+1},\quad
  \delta < \frac{\alpha'^2\epsilon}{\beta+\beta+1}
\]
we will have
\begin{align*}
  bb'-aa'&<(\beta + \beta' + 1)\delta < \epsilon\\
  \frac{b}{a'}-\frac{a}{b'}&<
  \frac{(\beta+\beta+1)\delta}{\alpha'^2} < \epsilon
\end{align*}

The set of number $aa'$, together with that of numbers $bb'$, and the set of numbers $\frac{a}{b'}$, together with that of numbers $\frac{b}{a'}$, define therefore two real numbers, which we will refer to as $cc'$ and $\frac{c}{c'}$.

The division so defined is the inverse of the multiplication. To establish this, we must establish the equality of the numbers $\frac{c}{c'}c=c_1$ and $c$, by showing that the rational numbers greater than one of them are greater than the other, and vice-versa.

Let $x$ be a number $>c_1$. It will be, by definition, greater than
% [Original: Page 7]
one of the numbers $\frac{b}{a'}b'$, and, a fortiori, greater than the number $b$; it will therefore be $> c$.

Reciprocally, if $x > c$, there will exist another rational number $x-\epsilon$ still greater than $c$, which means greater than one of the numbers $b$. Now we can determine, for any $\delta, a'$ and $b'$ such that $b'<a'+\delta$, that
\[
  \frac{b}{a'}b'<b+\frac{b\delta}{a'}<b+\frac{\beta\delta}{\alpha'}.
\]
If we choose $\delta$ less than $\frac{\alpha'\epsilon}{\beta'}$, we will have $x>\frac{b}{a'}b'$. Therefore $x > c_1$.

To extend the definition of the multiplication and the division to the case where one the factors, or both, are negative, we apply the rule of signs. To conclude we state without proof that a product is zero if one of the factors is.

We see, without trouble, that the operations so generalized, applied to rational numbers, will lead to the same results of the normal operations, and that the algebraic rules of computation remain valid without any change.

\paragraph{} We call \textit{absolute value} or \textit{module} of a real number $c$ the number itself, if it's positive, the number $-c$ if $c$ is negative. The module is often denoted by $|c|$. 

We clearly have
\begin{align*}
  |a||b| &= |ab| \\
  |a|-|b|-|c| &\leq a \pm b \pm c \leq |a|+|b|+|c|
\end{align*}

\paragraph{} Let $A,B$ be two sets of real numbers such that: first, any number in $B$ is greater than any number in $A$; second, we can find in $A$ and $B$ two numbers $a$ and $b$ such that $b-a<\epsilon$. Then there exists a unique real number $c$ such that we have
\[
  b\geq c \geq a
\]
for all $a$ in $A$ and $b$ in $B$.

In fact, let $A'$ be the set of rational numbers
% [Original: page 8]
less than at least one of the numbers in $A$; $B'$ the set of rational numbers greater than at least one of the numbers in $B$. The numbers in $B'$ will be greater than the ones in $A'$.

Moreover, take from $A$ and $B$ two numbers $a$ and $b$ such that $b-a<\frac \epsilon 3$. We can find two rational numbers around $a$ with a difference $<\frac \epsilon 3$ of which the smaller $a'$ belongs to $A'$, so we have
\[
  a - a' < \frac \epsilon 3.
\]
Likewise we can find in $B'$ a number $b'$ such that $b'-b$ is $<\frac\epsilon 3$. We have, therefore,
\[
  b' - a' < \epsilon.
\]

The sets $A',B'$, made of rational numbers, determine a real number $c$ greater than all the $a'$ and less than all the $b'$. This number satisfies the conditions required. In fact, if it's less than a number $a$ in $A$, then there would exist between $a$ and $c$ some rational number $a'$ greater than $c$, contrary to the definition of $c$. If it's greater than a number $b$ in $B$, we would arrive to an analogous contraddiction.

We can see, like in paragraph \ref{par:irrational-numbers}, that such a number $c$ is unique.

\paragraph{Limits} Let $x$ be a variable quantity, to which we give in succession an infinite sequence of values $x_1,\dots,x_n,\dots$. We say that the sequence $x_1,\dots,x_n,\dots$, or, in a more algebraic way, the variable $x$ \textit{tends} or \textit{converges} toward a \textit{limit} $c$ if, for all values of a positive quantity $\epsilon$, we can find an integer $\nu$ such that we have
\[
  |x_n-c| < \epsilon
\]
for all values of $n$ greater than $\nu$.

% [Original: page 9]

The variable $x$ cannot tend at the same time to two different limits $c$ and $c'$; because we would have
\[
  c'-c = (x_n-c)-(x_n-c'),
\]
from which
\[
  |c' - c| \leq |x_n-c| + |x_n-c'|.
\]
Therefore, regardless of $n$, at least one of the two modules
\[
  |x_n-c|,\quad|x_n-c'|
\]
would be at least equal to $\frac 1 2 |c'-c|$.

It is important to transform the preceeding definition, of a way to prove the existence of a limit, even when we are unable to determine it.

\paragraph{Theorem} \textit{
For a sequence $x_1,\dots,x_n,\dots$ to tend to a limit, it is necessary and sufficient that we can find a non-increasing sequence $\epsilon_1,\dots,\epsilon_n,\dots$ of positive numbers that has limit zero, and such that we have, for all values of the integers $n$ and $p$,
\[
  |x_n-x_{n+p}|<\epsilon_n
\]
}

\newcommand{\jseq}[2]{{#1}, \dots, {#2}, \dots}
$1^{st}$ Suppose, in fact, that the sequence $\jseq{x_1}{x_n}$ tends to a limit $c$. Let $\jseq{\delta_1}{\delta_m}$ be decreasing sequence of positive numbers with limit zero. We can, by hypothesis, find for some value of $m$ a whole number $\nu_m$ such that, for $n>\nu_m$,
\[
  |x_n-c|<\frac 1 2 \delta_m
\]
and, as a result,
\[
  |x_n-x_{n+p}| \leq |x_n - c| + |x_{n+p}-c|\leq\delta_m.
\]
If $n\leq \nu_1$, but $n+p>\nu_1$, on the other hand
\[
  |x_n-x_{n+p}| \leq |x_n - c| + |x_{n+p}-c|\leq e+\delta_m.
\]
where $e$ is greater than the quantities $\jseq{|x_1-c|}{|x_{\nu_1}-c|}$.

% [Original: page 10]

Finally, if both $n$ and $n+p$ are not $>\nu_1$, we will have
\[
  |x_n-x_{n+p}| \leq |x_n - c| + |x_{n+p}-c|\leq 2e.
\]

Define now a sequence of positive quantities $\jseq{\epsilon_1}{\epsilon_n}$ by the relations
\begin{align*}
  \epsilon_n &= 2e+\delta_1,\quad&\mbox{ if }n\leq \nu_1\\
  \epsilon_n &= \delta_m,\quad&\mbox{ if }\nu_m<n\leq<\nu_{m+1}\\
\end{align*}

We will have
\[
  |x_n-x_n+p|\leq \epsilon_n
\]
Moreover the quantities $\epsilon_n$ form a non-increasing sequence, and $\epsilon_n$ tends to zero when $n$ tends to $\infty$. Because we can find $m$ such that $\delta_m$ is less than a arbitrary quantity $\epsilon$, and it then suffices to take $n>\nu_m$ to be sure that
\[
  \epsilon_n\leq \delta_m<\epsilon.
\]

$2^{nd}$ Reciprocally, suppose that we could determine a sequence  $\jseq{\epsilon_1}{\epsilon_n}$ of positive numbers that tend to zero, and such that we have, at least for the values of $n$ that are greater than a fixed number $m$,
\begin{equation}\label{eqn:1}
  |x_n-x_m|<\epsilon_n
\end{equation}
and we want to show that the sequence $\jseq{x_1}{x_n}$ tends to a limit.

Denote with $a_\nu$ the greatest of the quantities
\[
  x_{m+1}-\epsilon_{m+1},\dots,x_\nu-\epsilon_\nu
\]
and with $b_\nu$ the smallest of the quantities
\[
  x_{m+1}+\epsilon_{m+1},\dots,x_\nu+\epsilon_\nu
\]
we clearly have
\[
  a_\nu\geq a_\mu\quad b_\nu\leq b_\mu\quad\mbox{if }\nu>\mu
\]

On the other hand, the inequality (\ref{eqn:1}) can be written as
\[
  -\epsilon_n<x_n-x_{n+p}<\epsilon_n
\]
from which
\[
  x_n-\epsilon_n<x_{n+p}<x_n+\epsilon_n
\]

Suppose $n\leq\nu$ and substitute in this equation $p$ with $p+\nu-n$; it becomes
\[
  x_n-\epsilon_n<x_{\nu+p}<x_n+\epsilon_n
\]
and as it occurs for the values $n=m_1,\dots\nu$, we deduce that
\[
  a_\nu<x_{\nu+p}<x_n+\epsilon_n.
\]

More generally for $\mu$ and $\nu$ any two positive integers $>m$, and $\lambda$ greater than them, we have, by a combination of the above inequalities
\[
  a_\nu\leq a_\lambda < b_\lambda < b_\nu
\]
So, any number $b$ is greater than any number $a$.

Moreover
\[
  b_\nu-a_\nu\leq(x_\nu+\epsilon_\nu) - (x_\nu-\epsilon_\nu)\leq 2\epsilon_\nu
\]
a quantity that can be made $<\epsilon$ by taking $\nu$ big enough.

Therefore the two sets of numbers $a$ and $b$ determine a number $c$. This number and the number $x_{\nu+p}$ being both comprised between $a_\nu$ and $b_\nu$, we have
\[
  |x_{\nu+p}-c|<\epsilon
\]
which is precisely the condition for the quantities $x$ to converge to $c$.

\paragraph{Corollary} \textit{
If the quantities $x_n$ are such that we always have
\[
  x_{n+1}\geq x_n
\]
they will converge to a limit $c$ or grow to eventually surpass any given quantity $E$.
}

Indeed, suppose first that, for any value
% [Original: page 12]
of the positive quantity $\delta$, we can find an integer $\nu$ such that we always have
\[
  x_{\nu+p}-x_\nu < \delta
\]
whatever is $p$.

Giving to $\delta$ a sequence of values $\delta_1,\delta_2\dots$ converging to zero; let $\nu_1,\nu_2,\dots$ be the corresponding values of $\nu$.

Moreover, let $n$ be a number $\geq\nu_k$ and $<\nu_{k+1}$. We have
\[
  |x_{n}-x_{n+p}|=x_{n+p} - x_n \leq x_{n+p} - x_{\nu_k}
\]
and if we set $n+p=\nu_k + q$,
\[
  |x_{n}-x_{n+p}|<x_{\nu_k + q} - x_n < \delta_k
\]

If we define the quantities $\jseq{\epsilon_{\nu_k}}{\epsilon_n}$ by the contition
\[
  \epsilon_n = \delta_k\quad\mbox{when }n\geq\nu_k<\nu_{k+1}
\]
we have
\[
  |x_{n}-x_{n+p}| < \epsilon_n.
\]

The quantities $\epsilon_n$ so defined, converge to zero, the quantities $x_n$ will tend to a limit.

Suppose, on the contrary, that there exists a quantity $\delta$ for which it's impossible to find a corresponding quantity $\nu$. We can, whatever is $n$, determine a number $n+p=n_1$ such that $x_{n_1}-x_n$ are all greater than $\delta$.
Therefore we can find a sequence of numbers $1, n_1, n_2,\dots,n_k,\dots$ such that we have
\[
  x_{n_1}-x_1\geq \delta,\quad x_{n_2}-x_{n_1}>\delta,\quad\dots
\]
from which
\[
  x_{n_k}>x_1+k\delta
\]

This number will become greater than any given number $E$ as soon as $k$ will be greater than $\frac{E-x_1}{\delta}$.

\paragraph{} \textit{If the variable $x$ tends to a limit $c$, $-x$ tends to the limit $-c$.}

% [Original: page 13]

This proposition is evident; because we have
\[
  |x-c|=|-x+c|.
\]
If for $n>\nu$
\[
  |x_\nu-c|<\epsilon,
\]
then at the same time
\[
  |-x_n+c|<\epsilon.
\]

\paragraph{} \textit{If the variable $x$ tends to a limit $c$ different from zero, $\frac 1 x$ will tend to the limit $\frac 1 c$.}

Let, in fact, $x_n-x=\xi_n$; we will have
\[
  \left|\frac{1}{x_n} - \frac 1 c\right| = 
    \left|-\frac{\xi_n}{c(c+\xi_n)}\right| =
    \frac{|\xi_n|}{|c||c+\xi_n|} \leq
    \frac{|\xi_n|}{|c|(|c|-|\xi_n|)}
\]
a quantity that will become $<\epsilon$ as soon as $n$ will be big enough to have
\[
  |\xi_n|\leq\frac{|c|^2\epsilon}{1+|c|\epsilon}.
\]

If $x$ tends to zero, we can, for any positive quantity $E$, find a number $\nu$ such that, for all values of $n$ greater than $\nu$,
\[
  |x_n|<\frac 1 E,
    \quad\mbox{from which}\quad
    \left|\frac 1 {x_n}\right|> E.
\]

Therefore $\frac 1 x$ doesn't tend to a any limit. We will agree, however, to say that it tends to $\infty$.

If, while tending to $\infty$, $x$ stays from a certain moment afterward constantly positive, we will say that it tends to $+\infty$. If it stays constantly negative, it will tend to $-\infty$.

\paragraph{} \textit{Let $x, y$ be two variable quantities that change simultaneously, and take respectively the sequences of values $x_1, y_1;\dots;x_n,y_n;\dots$. If $x, y$ tend respectively to 
% [Original: page 14]
the finite limits $c, d$, $x+y$ and $xy$ will tend to the limits $c+d$, $cd$.}

Let, in fact,
\[
  x_n-x=\xi_n,\quad y_n-d=\eta_n.
\]

We can, by definition, for any positive quantity $\delta$, find two quantities $\nu_1$ and $\nu_2$, such that
\begin{align*}
  |\xi_n|<\delta &\mbox{, if } n>\nu_1\\
  |\eta_n|<\delta &\mbox{, if } n>\nu_2
\end{align*}
and, consequently,
\[
  |\xi_n|<\delta,\quad|\eta_n|<\delta,\quad\mbox{if }n>\nu
\]
where $\nu$ is the maximum of $\nu_1$ and $\nu_2$.

Given this, we have
\begin{align*}
  |x_n+y_n-(x+d))| 
    &= |\xi_n+\eta_n\leq |\xi_n|+|\eta_n| < 2\delta,\\
  |x_ny_n-cd| &= |c\eta_n+d\xi_n+\xi_n\eta_n|\\
    &\leq |c||\eta_n|+|d||\xi_n| + |\xi_n||\eta_n|\\
    &<|c|\delta + |d|\delta + \delta^2
\end{align*}

The second memebers of these inequalities will be $<\epsilon$ if we take a $\delta$ such that at the same time
\[
  \delta <\frac\epsilon 2,\quad
    \delta < 1,\quad
    \delta < \frac{\epsilon}{|c|+|d|+1}.
\]

Our proposition has then been proved.

If $x$ tends to $\infty$ and $y$ continues to tend to finite limit $d$, we can, likewise, for arbitrary $E'$ and $\delta$, find a number $\nu$ such that
\[
  |x_n|>E',\quad
    |y_n-d|<\delta,\quad
    \mbox{if }n>\nu,
\]
from which
\begin{align*}
  |x_n-y_n| &= 
    |x_n+y_n-d+d|\geq
    |x_n| - |y_n-d| - |d|\\
    &> E'-\delta-|d| > E,
\end{align*}
if we choose $E'$ greater than $E+\delta+|d|$. Therefore, in this case, $x+y$ will tend to $\infty$.

% [Original: page 15]

The product $xy$ will tend to $\infty$ if $d$ is not zero. In fact, by choosing $\delta < |d|$ we will have
\[
  y_n = d + (y_n - d),
\]
from which
\[
  |y_n|\geq |d| - \delta,
\]
and finally
\[
  |x_ny_n| > E'(|d|-\delta) > E,
\]
if we choose $E'$ greater than $\frac{E}{|d|-\delta}$.

If $y$ tends to zero, and at the same time $x$ tends to $\infty$, we can't say anything \textit{a priori}.

Suppose finally that $x$ and $y$ both tend to $\infty$. We can't say say anything \textit{a priori} on the sum $x+y$. But the product will clearly tend to $\infty$.

\paragraph{} Combining the preceeding results, we can immediately deduce the following proposition:

\textit{Let $R(x,y,\dots)$ be any rational expression of the variables $x, y, \dots$. If these variables tend simultaneously to the limits $c, d, \dots$, $R(x,y,\dots)$ will tend to the limit $R(c,d,\dots)$.}

This theorem is however subject to the restriction that the operations involved in $R$, knowing $x, y,\dots$, can actually be performed for the particular values $x=c, y=d, \dots$.

If therefore among these operations there are divisions, it is necessary that the divisor is not zero.

\paragraph{} Arithmetics and Algebra have four fundamental operations: addition, subtraction, multiplication and division. We can conceive of a fifth, consisting of replacing a variable quantity with its limit. It's the introduction of this new operation that characterizes Infinitesimal Calculus.

% [Original: page 16]

\paragraph{Infinitesimals} We give the name \textit{infinitesimal} to all variable quantities that tend to zero; that of \textit{infinitely large} to all variable quantities that tend to $\infty$.

The inverse of an infinitesimal will therefore be infinitely large, and reciprocally.

The sequence of values $\jseq{x_1}{x_n}$, that we assign to an infinitesimal $x$, must have, by definition, zero as a limit; it cannot be subject to any other restriction. But we can also, if we want, subject it to other conditions: stipulate, for example, that these quantities will be positive, or rational, etc. Except for these restrictions, which will have to be specified in each case, we should consider them arbitrary.

\paragraph{} Two infinitesimals $x, y$ will be \textit{independent} if there exists no mandatory correlation between the values $\jseq{x_1}{x_n}$ and $\jseq{y_1}{y_n}$ respectively assigned to them.

On the contrary, if they are related in such a way that, knowing $x_n$, we can deduce $y_n$, we will say that $y$ is an infinitesimal \textit{dependent} on $x$. This dependence will be, in general, reciprocal, so that, knowing $y_n$, $x_n$ can be deduced.

Two infinitesimals $x, y$ being so related, we will say are of the same \textit{order} if, when $x$ tends to zero, $\frac y x$ has a constant quantity $c$ different from zero as a limit; $y$ will be of order greater than $x$, if $\frac y x$ has limit zero; and of lesser order, if $\frac y x$ tends to $\infty$.

We can, in the majority of cases, make this notion precise, and measure with a number the order of magnitude of an infinitesimal. We can say, in fact, that $y$ is of order $\alpha$ with respect to $x$, if the ratio $\frac{y}{x^\alpha}$ tends to a finite limit different from zero, when $x$ tends to zero.

% [Original: page 17]

According to this definition, a quantity $y$ that tends to a finite limit will be an infinitesimal of order zero; an infinitely large $y$ such that $yx^\alpha$ tends to a finite limit different from zero will be an infinitesimal of order $-\alpha$.

In all questions involving a number of infinitesimals $x,y,z,\dots$ dependent on each other, we can arbitrarily choose one of them as a unit of measure. This \textit{principal infinitesimal}, $x$, for example, being considered as having the unity as order of magnitude, $y, z, \dots$ will have orders of magnitude represented respectively by the numbers $\alpha, \beta, \dots$

The same comparison procedure is applicable to the infinitely large.

\paragraph{} Let $y$ be an infinitesimal of order $\alpha$; $A$ the limit to which $\frac{y}{x^\alpha}$ tends when $x$ tends to zero. We will have
\[
  \frac{y}{x^\alpha} = A + h,
  \quad\mbox{from which}\quad
  y = Ax^\alpha + hx^\alpha
\]
$h$ tending to zero with $x$.

The first term $Ax^\alpha$ is called the \textit{principal value} of $y$. It represents that infinitesimal having a relative error that decreases indefinitely with $x$.

If we are not satisfied with this first approximation, then we will have to determine the principal value of the remaining part. Let $Bx^\alpha$ be this principal value, we get a second value
\[
  y = Ax^\alpha + B x^\beta
\]
approaching at order $\beta$. In the same way we will find, if it's helpful, the principal value of the rest, and so on.

The determination of the principal values on an infinitesimal and the development in power series of the principal infinitesimal, which follows, is the main object of the First part of this Course.

% [Original: page 18]

The solution of this fundamental problem furnishes a precise method of approximation in all the applications of Mathematics; but that is not its only use: it allows to obtain entirely rigorous results, based on the following proposition.

\paragraph{} \textit{The ratio of two infinitesimals of the same order, $y$ and $z$, having principal values $Ax^\alpha$ and $Bx^\alpha$ respectively, has limit $\frac A B$.}

We have, in fact,
\[
  y=x^\alpha(A+h),\quad z = x^\alpha(B + k)
\]
$h$ and $k$ tending to zero with $x$. Therefore
\[
  \lim \frac x y = \lim\frac{A+h}{B+k} = \frac A B.
\]

\section{Topology}

\paragraph{} Let $x,y,\dots$ be variable quantities; we will call \textit{point} a set of simultaneous values $a, b,\dots$ given to these variables; the \textit{distance} of two points $p=(a,b,\dots)$ and $p' = (a',b',\dots)$ will be the expression
\[
  pp' = |a'-a| + |b'-b| + \dots
\]

We will say that the point $p=(a,b,\dots)$ is the \textit{limit} of a sequence of points
\[
  p_1=(x_1,y_1,\dots),
  \quad\dots,\quad
  p_n=(x_n,y_n,\dots),
\]
if the distance $pp_n$ has limit zero for $n$ that increases indefinitely, which amounts to say that $x_n,y_n,\dots$ have limits $a,b,\dots$ respectively.

% [Original: page 19]

By \textit{set} in this section we will mean a set of points, either finite or infinite. The \textit{dimension} of a set is the number of variables $x, y, \dots$ that define its points.

We call \textit{limit point} of a set $E$ any point that is the limit of a sequence of points in $E$ different from itself. The set of limit points is indicated by $E'$ and is called the \textit{derived set} of $E$.

Lets consider, for example, the case of a single dimension. From the preceding definitions, the set of integers has no limit point.

That of fractions $\frac 1 4,\frac 1 4, \frac 1 8,\dots$ has only the limit point $x=0$.

That of rational numbers $>a$ and $<b$ has for derived set the set of real numbers $\geq a$ and $\leq b$.

This last one is equal to its derived set.

We see from these examples that a set $E$ might contain points that are not in its derived $E'$, and vice versa.

If a point $p=(a,b,\dots)$ belongs to $E$ without being in $E'$, we can, by definition, find a quantity $\epsilon$ such that all other points of $E$ have a distance greater than $\epsilon$ from $p$. In this case we say that $p$ is an \textit{isolated point} of $E$.

\begin{thebibliography}{9}
\addcontentsline{toc}{chapter}{Bibliography}

\end{thebibliography}

\end{document}